\documentclass[a4paper, 14pt]{extreport}
\usepackage[utf8x]{inputenc}
\usepackage[russian]{babel}
\usepackage{graphicx}
\usepackage{amsmath}
\usepackage{import}
\usepackage{xifthen}
\usepackage{pdfpages}
\usepackage{transparent}
\usepackage{calc}
\usepackage{setspace}
\usepackage{geometry}
\usepackage{titlesec}
\usepackage{listings}
\usepackage{color}
\usepackage{verbatim}
\usepackage[unicode, pdftex]{hyperref}
\geometry{a4paper, portrait, margin=15mm, bmargin=15mm, tmargin=20mm}
\graphicspath{ {./figure/} }

\titleformat{\section}[block]{\normalfont\Large\bfseries\centering}{\thesection}{1ex}{}

\newcommand{\incfig}[1]{%
    \def\svgheight{\swgwidth}
    \import{./figure/}{#1.pdf_tex}
}

\newcommand{\anonchapter}[1]{\chapter*{#1}
\addcontentsline{toc}{chapter}{#1}
}


\definecolor{dkgreen}{rgb}{0,0.6,0}
\definecolor{gray}{rgb}{0.5,0.5,0.5}
\definecolor{mauve}{rgb}{0.58,0,0.82}

\lstset{ %
language=C,                 % выбор языка для подсветки (здесь это С)
basicstyle=\small\tffamily, % размер и начертание шрифта для подсветки кода
numbers=left,               % где поставить нумерацию строк (слева\справа)
numberstyle=\tiny,           % размер шрифта для номеров строк
stepnumber=1,                   % размер шага между двумя номерами строк
numbersep=5pt,                % как далеко отстоят номера строк от подсвечиваемого кода
backgroundcolor=\color{white}, % цвет фона подсветки - используем \usepackage{color}
showspaces=false,            % показывать или нет пробелы специальными отступами
showstringspaces=false,      % показывать или нет пробелы в строках
showtabs=false,             % показывать или нет табуляцию в строках
frame=single,              % рисовать рамку вокруг кода
tabsize=2,                 % размер табуляции по умолчанию равен 2 пробелам
captionpos=b,              % позиция заголовка вверху [t] или внизу [b] 
breaklines=true,           % автоматически переносить строки (да\нет)
breakatwhitespace=false, % переносить строки только если есть пробел
escapeinside={\%*}{*)}   % если нужно добавить комментарии в коде
}

\begin{document}
\chapter{Постановка задачи}
\par Жизнь в 21 веке тесно связана с технологиями. Каждый второй человек сейчас имеет полную по Тьюрингу вычислительную машину, которую называют телефоном. Все эти технологические новшества помогают сделать нашу и без того комфортную жизнь, еще более комфортной. Следующей вехой развития я вижу развитие IoT ( интернет - вещей ).
\section{IoT}
\par Термин IoT, или Интернет вещей, относится к коллективной сети подключенных устройств и технологии, которая облегчает связь между устройствами и облаком, а также между самими устройствами. Благодаря появлению недорогих компьютерных микросхем и телекоммуникаций с высокой пропускной способностью у нас теперь есть миллиарды устройств, подключенных к Интернету. Это означает, что повседневные устройства, такие как зубные щетки, пылесосы, автомобили и механические установки, могут использовать датчики для сбора данных и разумного реагирования на действия пользователей.
\section{Cloud robotics}
\par В связи с развитием технологий передачи информации на большие растояния с большой скоростью, многие робототехнические компании заинтересовались IoT и тем, как можно скрестить их с робототехникой в целом. Такой симбиоз получился очень удачным и открыл новую область робототехники - cloud robotics ( облачная робототехника ).
\section{Цель работы}
\par Целью мое работы будет разработка ПО для взаимодействия робота с некоторой облачной инфраструктурой, для задач SLAM алгоритма. То есть задача SLAM будет решаться в облаке.
\par Возникает резонный вопрос, зачем переносить SLAM в облако? Вот несколько ключевых причин:
\begin{itemize}
        \item Снижение загруженности CPU/GPU на бортовом компьютере робота
        \item Снижение зависимости робота от локально построенной карты помещения
        \item Снижение потребления заряда батареи
        \item Снижение стоимости робота
\end{itemize}
\chapter{Обзор состояния науки и техники}
\par Для начала выделим необходимые технология для обеспечения работоспособности Cloud SLAM:
\begin{itemize}
        \item Стабильная сеть с высокой пропускной способностью
        \item Платформы для разработки топологии взаимодействия между компонентами SLAM, которые имеют низкий overhead
        \item Технологии для хранения данных, которые имеют низкий overhead
        \item Технологии для передачи потоковых данных, которые имеют низкий overhead
\end{itemize}
\par Из этих абстрактных требований мы можем понять, что для обечпечения жизнеспособности такого метода вычисления карты и положения на ней требуются высокий уровень технического обеспечения и знаний. Такие жесткие требования проистекают от требования в низкой задержке. Если SLAM алгоритм имеет высокую задержку, то он оказывается бесполезным для работы.
\section{Обзор литературы}
\par Мной был проведен поиск литературы по схожей проблеме. Вот несколько статей:
\begin{itemize}
        \item Supun Kamburugamuve, Leif Christiansen, Geoffrey Fox. - A Framework for Real Time Processing of Sensor Data in Cloud. - Indiana University:April 2015. - 12 p.
        \item Supun Kamburugamuve, Hegjing He, Geoffrey Fox. - Cloud-based Parallel Implementation of SLAM for Mobile Robots. - Indiana University:December 2017. - 8 p.
\end{itemize}
\par Обе эти статьи описывают типичное решения для задачи SLAM в облаке. В моей работе я хочу провести расширение идей вычислений в облаке для задач картографирования и локализации на карте.
\chapter{Реализация}
\par Архитектура подобного приложения предполагает 3 слоя:
\begin{enumerate}
        \item Gateway layer
        \item Message Broker layer
        \item Processing layer
\end{enumerate}
\par На Gateway layer мы предпологаем некое приложение или уже готовую технологию для связи драйвера робота и Message Broker - ов, таких как Kafka или RabbitMQ. Типичные решения - отсутсвуют(ну или я не нашел).
\par На Message Broker layer мы распологаем брокеров сообщении, которые будут выступать в роли очереди для скопившихся сообщений, которые еще не были обработаны на Processing layer. Типичное решение - kafka, RabbitMQ
\par На Processing layer распологается сама логика. В нашем случае - SLAM. На картинке выше представлен SLAM на основе фильтра частиц, который хорошо параллелиться, чего нельзя сказать о графовых SLAM. Типичное решение - Apache Storm из-за его ориентацию на stream ( те на потоки данных ), а не на batch ( пакеты ) данных.
\chapter{Apache Storm}
\par Для реализации топологии Apache Storm ( на 3-ем уровне ) есть моменты которые нужно учитывать. В данном фреймворке всем управляет Nimbus, он занимается распределением задач и сериализацией потоков данных в системе. Однако, управление идет не от него напрямую, а черезе специальную ноду - Zookeeper, которая управляет кластером машин ( может и виртуальных ), на которых запущенна сама топология - Spout ( принимающие и передающие tuple-ам информацию ) и Bolt ( принимающие, обрабатывающие и передающие tuple информацию ) нод. Визуально это выглядит следующим образом:
\chapter{Обзор традиционных решений}
\par Cloud robotics, к сожалению, пока всего лишь тенденция. SLAM прочно закрепился в inter ( внутреннем ) взаимодействии, к тому же, многие методы SLAM хорошо вписываются в архитектуру CPU, за счет чего, становяться более оптимизированными.
\par Однако, в традиционном использовании SLAM алгоритмов мы ограничены производительностью бортового компьютера робота, в то время как на сервере, мы можем создать целый кластер компьютеров, которые в десятки раз быстрее буду обрабатывать данные с сенсоров. Хотелось бы сказать, что SLAM алгоритмы, которые используются на сервере и на роботах, по логике работы ничем не отличаются, поэтому ,чтобы выбрать наиболее подходящий нам SLAM алгоритм, мы проведем обзор.
\par Сравнивать будем по следующим критериям:
\begin{enumerate}
        \item Используемые датчики, их максимальное количество.
        \item Performance построения карты.
        \item Performance локализации.
        \item Accuracy построения карты.
        \item Accuracy локализации.
        \item Совместимость с пакетом robot\_localization.
\end{enumerate}
\par Performance построения карты сложно оценить полностью независимо от локализации. Поэтому, за оценку этого критерия будем брать загруженность CPU компьютера со следующими характеристиками:
\begin{itemize}
        \item CPU: AMD Ryzen 7 4800H \ 8 kernel \ 2.9 GHz ( 4.2 GHz - Turbo )
        \item Mem: 16 GB DDR4 3200
        \item GPU: NVIDIA GeForce GTX 1650 Ti \ 4 Gb video memory
\end{itemize}
\par Чтобы оценить performance локализации, будем замерять время от старта локализации, до того момента, пока возможные положения и ориентация робота не предут в стабильное состояние. Те будем замерять время на локализацию. При этом, для чистоты эксперемента, будем локализироваться по одной и той же траектории.
\par Дабы оценить Accuracy построения карты, будем сравнивать построенную алгоритмом карту и карту помещения построенную в CAD системе.
\par Accuracy локализации можно замерить встав на некоторую точку с заранее известными координатами. Затем провести сравнение координат, полученных из tf топиков и заранее известных координат.
\par Так же для этих целей, у меня есть bag файл, любезно записанный моими коллегами. Будем использовать его. Скачать bag файл можно на моем github: \url{https://github.com/sees1/MagMPEI.git} 
\section{Обзор SLAM}
\subsection{OpenSlam's Gmapping + AMCL}
\par slam\_gmapping - первая часть laser-base SLAM, который отвечает за mapping. Предположительно на фильтре частиц. Хорошо поддерживается, много документации, много примеров в открытом доступе.
\par\noindent AMCL - вторая часть laser-base SLAM, которая отвечает за localization. Основан на фильтре частиц. Хорошо поддерживается, много документации, много примеров в открытом доступе.
\vspace{2mm}
\par\noindent \underline{Характеристики:}
\par\textbf{Используемые датчики, их максимальное количество:} odometry(1 unit) + laser(1 unit)
\par\textbf{Performance построение карты:} -
\par\textbf{Performance локализации:} -
\par\textbf{Accuracy построение карты:} -
\par\textbf{Accuracy локализации:} -
\par\textbf{Совместимость с пакетом RL:} совместим
\par\textbf{slam\_gmapping repo:} \hspace{1mm}\url{https://github.com/ros-perception/slam_gmapping}
\par\textbf{AMCL repo:}\hspace{1mm}\url{https://github.com/ros-planning/navigation/tree/noetic-devel/amcl}
\subsection{hector\_slam}
\par Полноценный laser-base SLAM. Предположительно на фильтре частиц. Динамически строит карту ( отправляет ее в топик ) и динамически локализируется на строящейся карте.
\par\noindent Есть способы пробросить динамическую карту в move\_base. Использование локализаций из amcl не нужно, потому что hector при построении, автоматом локализуется на карте.
\par\noindent Хорошо поддерживается.
\vspace{2mm}
\par\noindent \underline{Характеристики:}
\par\textbf{Используемые датчики, их максимальное количество:} laser(1 unit) + Optional:odometry(1 unit)
\par\textbf{Performance построение карты:} -
\par\textbf{Performance локализации:} -
\par\textbf{Accuracy построение карты:} -
\par\textbf{Accuracy локализации:} -
\par\textbf{Совместимость с пакетом RL:} совместим
\par\textbf{repo:}\url{https://github.com/tu-darmstadt-ros-pkg/hector_slam}
\subsection{rtabmap\_ros}
\par SLAM работающий почти со всеми типами sensor\_msgs. Основан на графах. Хорошо поддерживается, много документации.
\par\noindent Можно использовать, как самостоятельно решение, без move\_base пакета, для планирования, т.к. есть еще и Global/Local Planner.
\vspace{2mm}
\par\noindent \underline{Характеристики:}
\par\textbf{Используемые датчики, их максимальное количество:} laser(1 unit)/lidar(1 unit)/camera(1 unit) + Optional:odometry(1 unit) + Optional:IMU(1 unit) + Optional:GPS(1 unit)
\par\textbf{Performance построение карты:} -
\par\textbf{Performance локализации:} -
\par\textbf{Accuracy построение карты:} -
\par\textbf{Accuracy локализации:} -
\par\textbf{Совместимость с пакетом RL:} совместим
\par\textbf{repo:}\url{https://github.com/introlab/rtabmap_ros}
\subsection{cartographer\_ros}
\par SLAM работающий в основном с laserScan/PointCloud. Основан на графах. Есть возможность использовать multiSLAM. Хорошо поддерживается, много документации.
\par\noindent Можно использовать, как самостоятельно решение, без move\_base пакета, для планирования, т.к. есть еще и Global/Local Planner.
\vspace{2mm}
\par\noindent \underline{Характеристики:}
\par\textbf{Используемые датчики, их максимальное количество:} laser(1 unit)/lidar(1 unit)/camera(1 unit) + Optional:odometry(1 unit) + Optional:IMU(1 unit) + Optional:GPS(1 unit)
\par\textbf{Performance построение карты:} -
\par\textbf{Performance локализации:} -
\par\textbf{Accuracy построение карты:} -
\par\textbf{Accuracy локализации:} -
\par\textbf{Совместимость с пакетом RL:} совместим
\par\textbf{repo:}\url{https://github.com/cartographer-project/cartographer_ros}
\subsection{slam\_karto}
\par SLAM работающий с laserScan. Предположительно основан на графах. Последний коммит 2020 года
\vspace{2mm}
\par\noindent \underline{Характеристики:}
\par\textbf{Используемые датчики, их максимальное количество:} laser(1 unit) + Optional:odometry(1 unit)
\par\textbf{Performance построение карты:} -
\par\textbf{Performance локализации:} -
\par\textbf{Accuracy построение карты:} -
\par\textbf{Accuracy локализации:} -
\par\textbf{Совместимость с пакетом RL:} совместим
\par\textbf{repo:}\url{https://github.com/ros-perception/slam_karto}
\subsection{mrpt\_slam}
\par Обертка над SLAM алгоритмами из библиотеки Mobile Robot Programming Toolkit (MRPT). Есть как EKF алгоритмы SLAM, так и RBPF SLAM, Graph SLAM. Последний коммит 2023 года ( хорошо поддерживается )
\vspace{2mm}
\par\noindent \underline{Характеристики:}
\par\textbf{Используемые датчики, их максимальное количество:} Probably:laser(1 unit)/lidar(1 unit) + Optional:odometry(1 unit)
\par\textbf{Performance построение карты:} -
\par\textbf{Performance локализации:} -
\par\textbf{Accuracy построение карты:} -
\par\textbf{Accuracy локализации:} -
\par\textbf{Совместимость с пакетом RL:} совместим
\par\textbf{repo:}\url{https://github.com/mrpt-ros-pkg/mrpt_slam}
\subsection{rgbd\_slam}
\par SLAM работающий с rgbd камерами, картинка с которых преобразуется в плотное облако точек и используется в качестве карты. Предположительно основан на фильтре цастиц. Последний коммит 2018 года.
\vspace{2mm}
\par\noindent \underline{Характеристики:}
\par\textbf{Используемые датчики, их максимальное количество:} Probably:lidar(1 unit) or rgbdCam(1 unit) + odometry(1 unit)
\par\textbf{Performance построение карты:} -
\par\textbf{Performance локализации:} -
\par\textbf{Accuracy построение карты:} -
\par\textbf{Accuracy локализации:} -
\par\textbf{Совместимость с пакетом RL:} совместим
\par\textbf{repo:}\url{https://github.com/felixendres/rgbdslam_v2}
\subsection{ohm\_tsd\_slam}
\par SLAM работающий с laserScan. Основан на TSDF (truncated signed distance transform) алгоритме. Последний коммит 2020 года.
\vspace{2mm}
\par\noindent \underline{Характеристики:}
\par\textbf{Используемые датчики, их максимальное количество:} laser(1 unit) + Optional:odometry(1 unit)
\par\textbf{Performance построение карты:} -
\par\textbf{Performance локализации:} -
\par\textbf{Accuracy построение карты:} -
\par\textbf{Accuracy локализации:} -
\par\textbf{Совместимость с пакетом RL:} совместим
\par\textbf{repo:}\url{https://github.com/autonohm/ohm_tsd_slam}
\subsection{tiny\_slam\_ros\_cpp}
\par SLAM работающий с laserScan. Предположительно основан на фильтре частиц. Последний коммит 2017 года.
\vspace{2mm}
\par\noindent \underline{Характеристики:}
\par\textbf{Используемые датчики, их максимальное количество:} laser(1 unit) + Optional:odometry(1 unit)
\par\textbf{Performance построение карты:} -
\par\textbf{Performance локализации:} -
\par\textbf{Accuracy построение карты:} -
\par\textbf{Accuracy локализации:} -
\par\textbf{Совместимость с пакетом RL:} совместим
\par\textbf{repo:}\url{https://github.com/OSLL/tiny-slam-ros-cpp}
\subsection{stereo\_slam}
\par SLAM работающий со stereo cam. Предположительно основан на графах. Последний коммит 2017 года.
\vspace{2mm}
\par\noindent \underline{Характеристики:}
\par\textbf{Используемые датчики, их максимальное количество:} stereo camera(1 unit) + Optional:odometry(1 unit)
\par\textbf{Performance построение карты:} -
\par\textbf{Performance локализации:} -
\par\textbf{Accuracy построение карты:} -
\par\textbf{Accuracy локализации:} -
\par\textbf{Совместимость с пакетом RL:} совместим
\par\textbf{repo:}\url{https://github.com/srv/stereo_slam}
\subsection{lsd\_slam}
\par SLAM работающий со mono cam. Предположительно основан на графах. Последний коммит 2014 года.
\vspace{2mm}
\par\noindent \underline{Характеристики:}
\par\textbf{Используемые датчики, их максимальное количество:} mono camera(1 unit) + Optional:odometry(1 unit)
\par\textbf{Performance построение карты:} -
\par\textbf{Performance локализации:} -
\par\textbf{Accuracy построение карты:} -
\par\textbf{Accuracy локализации:} -
\par\textbf{Совместимость с пакетом RL:} совместим
\par\textbf{repo:}\url{https://github.com/tum-vision/lsd_slam}
\subsection{crsm\_slam}
\par SLAM работающий с laserScan. Предположительно основан на фильтре частиц. Последний коммит 2016 года.
\vspace{2mm}
\par\noindent \underline{Характеристики:}
\par\textbf{Используемые датчики, их максимальное количество:} laser(1 unit) + Optional:odometry(1 unit)
\par\textbf{Performance построение карты:} -
\par\textbf{Performance локализации:} -
\par\textbf{Accuracy построение карты:} -
\par\textbf{Accuracy локализации:} -
\par\textbf{Совместимость с пакетом RL:} совместим
\par\textbf{repo:}\url{https://github.com/etsardou/crsm-slam-ros-pkg/tree/indigo-devel}
\subsection{vslam}
\par Очень мало информации.
\par\textbf{repo:}\url{https://code.ros.org/svn/ros-pkg/stacks/vslam/trunk}
\end{document}
